% !TEX program = xelatex % (can also use "lualatex") 
\documentclass{beamer}
\usepackage{fontspec}
\usepackage{polyglossia}
\setmainlanguage{russian}
\setotherlanguage{english}

% Настройка шрифтов
\setmainfont{CMU Sans Serif}
\setsansfont{CMU Sans Serif}
\setmonofont{CMU Typewriter Text}

% Дополнительные пакеты
\usepackage{graphicx}
\usepackage{listings}
\usepackage{xcolor}
\usepackage{amsmath}
\usepackage{caption} % Добавляем пакет для работы с подписями

% Цветовая схема для кода
\lstset{
    basicstyle=\ttfamily\small,
    keywordstyle=\color{blue},
    commentstyle=\color{green!60!black},
    stringstyle=\color{red},
    showstringspaces=false,
    numbers=left,
    numberstyle=\tiny\color{gray}
}

% Тема презентации
\usetheme{Berlin}
\title{Построение фрактала «Кривая Коха» с помощью Python}
\author{Мельник Наталья Олеговна}
\date{\today}
\institute{РГПУ им. А.И. Герцена}

\begin{document}

% Титульный слайд
\begin{frame}
    \titlepage
\end{frame}

% Содержание
\begin{frame}{Содержание}
    \tableofcontents
\end{frame}

\section{Введение}
\begin{frame}{Что такое фрактал Коха?}
    \begin{block}{История}
        Кривая Коха — фрактал, описанный шведским математиком Хельге фон Кохом в 1904 году.
    \end{block}
    
    \begin{itemize}
        \item<1-> Простой геометрический фрактал
        \item<2-> Бесконечная длина при конечной площади
        \item<3-> Самоподобная структура
        \item<4-> Применение в компьютерной графике
    \end{itemize}
    
    \uncover<5>{
        \begin{alertblock}{Интересный факт}
            Кривая Коха является непрерывной, но нигде не дифференцируемой!
        \end{alertblock}
    }
\end{frame}

\begin{frame}{Геометрическое построение}
    \begin{columns}
        \begin{column}{0.5\textwidth}
            \begin{enumerate}
                \item<1-> Начальный отрезок
                \item<2-> Разделить на 3 части
                \item<3-> Построить равносторонний треугольник
                \item<4-> Удалить основание треугольника
                \item<5-> Повторить для каждого отрезка
            \end{enumerate}
        \end{column}
        \begin{column}{0.5\textwidth}
            \begin{figure}
                \centering
                \includegraphics[width=0.9\textwidth]{i.png}
                \caption{Процесс построения кривой Коха}
            \end{figure}
        \end{column}
    \end{columns}
\end{frame}

\section{Математические основы}
\begin{frame}{Математическое описание}
    \begin{block}{Рекуррентное соотношение}
        На каждом шаге длина кривой увеличивается в $\frac{4}{3}$ раза:
        $$L_n = L_0 \times \left(\frac{4}{3}\right)^n$$
    \end{block}
    
    \begin{exampleblock}{Размерность Хаусдорфа}
        Размерность фрактала Коха вычисляется по формуле:
        $$D = \frac{\log 4}{\log 3} \approx 1.26186$$
    \end{exampleblock}
    
    \begin{itemize}
        \item Бесконечная длина при $n \to \infty$
        \item Конечная площадь ограниченной фигуры
    \end{itemize}
\end{frame}

\section{Реализация на Python}
\begin{frame}[fragile]{Базовый алгоритм}
    \begin{lstlisting}[language=Python, caption=Рекурсивная функция построения]
import turtle
import math

def koch_curve(t, length, depth):
    if depth == 0:
        t.forward(length)
    else:
        length /= 3.0
        koch_curve(t, length, depth-1)
        t.left(60)
        koch_curve(t, length, depth-1)
        t.right(120)
        koch_curve(t, length, depth-1)
        t.left(60)
        koch_curve(t, length, depth-1)
    \end{lstlisting}
\end{frame}

\begin{frame}[fragile]{Полная реализация}
    \begin{lstlisting}[language=Python, caption=Построение снежинки Коха]
def koch_snowflake(t, length, depth):
    for i in range(3):
        koch_curve(t, length, depth)
        t.right(120)

# Настройка отображения
screen = turtle.Screen()
screen.setup(800, 600)
t = turtle.Turtle()
t.speed(0)

# Построение снежинки
koch_snowflake(t, 300, 4)
screen.exitonclick()
    \end{lstlisting}
\end{frame}

\section{Результаты}
\begin{frame}{Визуализация результатов}
    \begin{columns}
        \begin{column}{0.5\textwidth}
            \begin{figure}
                \centering
                \includegraphics[width=0.9\textwidth]{s.png}
                \caption{Кривая Коха (n=4)}
            \end{figure}
        \end{column}
        \begin{column}{0.5\textwidth}
            \begin{figure}
                \centering
                \includegraphics[width=0.9\textwidth]{k.png}
                \caption{Снежинка Коха (n=5)}
            \end{figure}
        \end{column}
    \end{columns}
\end{frame}

\begin{frame}{Сравнительный анализ}
    \begin{table}
        \centering
        \begin{tabular}{lccc}
            \hline
            \textbf{Глубина} & \textbf{Кол-во отрезков} & \textbf{Длина} & \textbf{Время (с)} \\
            \hline
            n=0 & 1 & 1.00 & 0.001 \\
            n=1 & 4 & 1.33 & 0.005 \\
            n=2 & 16 & 1.78 & 0.021 \\
            n=3 & 64 & 2.37 & 0.085 \\
            n=4 & 256 & 3.16 & 0.341 \\
            n=5 & 1024 & 4.21 & 1.364 \\
            \hline
        \end{tabular}
        \caption{Зависимость параметров от глубины рекурсии}
    \end{table}
\end{frame}

\section{Применение и выводы}
\begin{frame}{Практическое применение}
    \begin{block}{Компьютерная графика}
        \begin{itemize}
            \item Генерация природных ландшафтов
            \item Создание текстур
            \item Моделирование сложных форм
        \end{itemize}
    \end{block}
    
    \begin{block}{Образование}
        \begin{itemize}
            \item Изучение рекурсии
            \item Визуализация математических концепций
            \item Развитие алгоритмического мышления
        \end{itemize}
    \end{block}
\end{frame}

\begin{frame}{Выводы}
    \begin{block}{Достигнутые результаты}
        \begin{itemize}
            \item Реализован алгоритм построения кривой Коха на Python
            \item Исследованы математические свойства фрактала
            \item Визуализированы различные итерации
            \item Проанализирована вычислительная сложность
        \end{itemize}
    \end{block}
    
    \begin{alertblock}{Перспективы развития}
        \begin{itemize}
            \item Оптимизация алгоритма
            \item 3D-версия фрактала
            \item Интерактивная визуализация
        \end{itemize}
    \end{alertblock}
\end{frame}

\begin{frame}
    \begin{center}
        \Huge \textbf{Спасибо за внимание!}
        
    \end{center}
\end{frame}

\end{document}