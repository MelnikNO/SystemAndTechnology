\documentclass[a0,portrait]{tikzposter}
\usepackage[utf8]{inputenc}
\usepackage[T2A]{fontenc}
\usepackage[russian]{babel}
\usepackage{amsmath}
\usepackage{amsfonts}
\usepackage{amssymb}
\usepackage{graphicx}
\usepackage{tikz}
\usepackage{listings}
\usepackage{xcolor}
\usetikzlibrary{shapes,arrows,positioning}

% Настройка темы и цветов
\usetheme{Simple}
\usecolorstyle{Germany}

% Убираем стандартный заголовок
\title{}
\author{}
\institute{}
\date{}

\begin{document}

% Создаем кастомный заголовок по центру
\begin{center}
\vspace{2cm}
{\Huge\textbf{Мельник Наталья Олеговна}} \\
\vspace{1cm}
{\LARGE Построение фрактала «Кривая Коха» с помощью Python} \\
\vspace{1cm}
{\large Российский государственный педагогический университет им. А.И. Герцена} \\
\vspace{0.5cm}
{\large \today}
\end{center}

\vspace{1cm}

\begin{columns}
    % Левая колонка
    \column{0.33}
    
    \block{Аннотация}{
        Исследование посвящено разработке алгоритма построения фрактала «Кривая Коха» на языке Python. Рассмотрены математические основы и практическая реализация с использованием рекурсивных методов.
        
        \textbf{Ключевые слова:} фракталы, кривая Коха, Python, рекурсия, геометрические преобразования.
    }
    
    \block{Введение}{
        \textbf{Фракталы} — геометрические объекты с бесконечно самоподобной структурой, сохраняющейся при любом масштабе.
        
        \textbf{Кривая Коха} — классический фрактал, описанный шведским математиком Хельге фон Кохом в 1904 году. Является непрерывной, но нигде не дифференцируемой кривой.
        
        \begin{tikzfigure}
            \begin{tikzpicture}[scale=0.8]
                % Итерации построения кривой Коха
                \draw[thick, blue] (0,0) -- (3,0);
                \node at (1.5,-0.3) {n=0};
                
                \draw[thick, red] (4,0) -- (5,0) -- (5.5,0.866) -- (6,0) -- (7,0);
                \node at (5.5,-0.3) {n=1};
                
                \draw[thick, green] (8,0) -- (8.67,0) -- (9,0.288) -- (9.33,0) -- (9.67,0.288) -- 
                (10,0) -- (10.33,0.288) -- (10.67,0) -- (11,0);
                \node at (9.5,-0.3) {n=2};
            \end{tikzpicture}
            \caption{Итерационное построение кривой Коха}
        \end{tikzfigure}
    }
    
    \block{Математические основы}{
        \textbf{Основные формулы:}
        
        Длина кривой на n-ой итерации: 
        $$L_n = L_0 \times \left(\frac{4}{3}\right)^n$$
        
        Количество элементарных отрезков:
        $$N_n = 4^n$$
        
        Фрактальная размерность (размерность Хаусдорфа):
        $$D = \frac{\log 4}{\log 3} \approx 1.26186$$
        
        \textbf{Геометрическое построение:}
        Каждый отрезок делится на 3 равные части, средняя часть заменяется на два отрезка длиной 1/3, образующих равносторонний треугольник.
    }

    % Центральная колонка
    \column{0.34}
    
    \block{Алгоритм построения}{
        \textbf{Рекурсивный подход:}
        
        \begin{lstlisting}[basicstyle=\scriptsize, language=Python, backgroundcolor=\color{lightgray}]
import turtle

def koch_curve(t, length, depth):
    """
    Рекурсивная функция построения кривой Коха
    t - объект черепахи
    length - длина отрезка
    depth - глубина рекурсии
    """
    if depth == 0:
        t.forward(length)
    else:
        # Делим отрезок на 3 части
        length /= 3.0
        # Рекурсивно строим 4 отрезка
        koch_curve(t, length, depth-1)
        t.left(60)      # Поворот на 60° влево
        koch_curve(t, length, depth-1)
        t.right(120)    # Поворот на 120° вправо
        koch_curve(t, length, depth-1)
        t.left(60)      # Поворот на 60° влево
        koch_curve(t, length, depth-1)

# Создание снежинки Коха
def koch_snowflake(t, length, depth):
    for _ in range(3):
        koch_curve(t, length, depth)
        t.right(120)
        \end{lstlisting}
    }
    
    \block{Методы исследования}{
        \textbf{Инструменты и технологии:}
        \begin{itemize}
            \item Python 3.8+
            \item Turtle Graphics для визуализации
            \item Matplotlib для анализа данных
            \item Time module для измерения производительности
        \end{itemize}
        
        \textbf{Методика исследования:}
        \begin{itemize}
            \item Анализ рекурсивных алгоритмов
            \item Исследование вычислительной сложности
            \item Визуализация геометрических преобразований
            \item Сравнение теоретических и практических результатов
        \end{itemize}
    }

    % Правая колонка
    \column{0.33}
    
    \block{Результаты и визуализация}{
        \textbf{Численные характеристики:}
        
        \begin{center}
            \begin{tabular}{|c|c|c|c|}
                \hline
                \textbf{n} & \textbf{Отрезки} & \textbf{Длина} & \textbf{Время (с)} \\
                \hline
                0 & 1 & 1.00 & 0.001 \\
                1 & 4 & 1.33 & 0.005 \\
                2 & 16 & 1.78 & 0.021 \\
                3 & 64 & 2.37 & 0.085 \\
                4 & 256 & 3.16 & 0.341 \\
                5 & 1024 & 4.21 & 1.364 \\
                \hline
            \end{tabular}
        \end{center}
        
        \begin{tikzfigure}
            \begin{tikzpicture}[scale=0.5]
                % Снежинка Коха n=2
                \begin{scope}[rotate=0]
                    \draw[thick,blue] (0,0) -- (2,0);
                    \draw[thick,blue] (2,0) -- (1,1.732);
                    \draw[thick,blue] (1,1.732) -- (0,0);
                \end{scope}
                \begin{scope}[shift={(3,0)}]
                    \draw[thick,red] (0,0) -- (0.67,0) -- (1,0.288) -- (1.33,0) -- (2,0);
                    \draw[thick,red] (2,0) -- (1.67,0.288) -- (1.33,0) -- (1,0.866) -- (0.67,0.288) -- (0,0);
                    \draw[thick,red] (0,0) -- (0.33,0.577) -- (0.67,0.288) -- (1,0.866) -- (1.33,0.577) -- (2,0);
                \end{scope}
                \node at (1,-1) {n=1};
                \node at (4,-1) {n=2};
            \end{tikzpicture}
            \caption{Снежинка Коха на разных итерациях}
        \end{tikzfigure}
        
        \textbf{Диаграмма роста длины:}
        \begin{tikzfigure}
            \begin{tikzpicture}[scale=0.6]
                \draw[->] (0,0) -- (6,0) node[right] {n (итерация)};
                \draw[->] (0,0) -- (0,5) node[above] {L(n) (длина)};
                \foreach \y in {1,2,3,4,5} 
                    \draw (0.1,\y) -- (-0.1,\y) node[left] {\y};
                \foreach \x in {1,2,3,4,5} 
                    \draw (\x,0.1) -- (\x,-0.1) node[below] {\x};
                
                \draw[thick,blue] (0,1) -- (1,1.33) -- (2,1.78) -- (3,2.37) -- (4,3.16) -- (5,4.21);
                \node[blue] at (3,3.5) {Экспоненциальный рост длины};
                
                % Точки на графике
                \foreach \x/\y in {0/1,1/1.33,2/1.78,3/2.37,4/3.16,5/4.21}
                    \fill[red] (\x,\y) circle (2pt);
            \end{tikzpicture}
            \caption{Зависимость длины кривой от номера итерации}
        \end{tikzfigure}
    }
    
    \block{Выводы}{
        \textbf{Основные результаты:}
        \begin{itemize}
            \item Успешно реализован рекурсивный алгоритм построения кривой Коха
            \item Исследованы фрактальные свойства кривой (самоподобие, размерность)
            \item Проанализирована вычислительная сложность алгоритма: $O(4^n)$
            \item Подтверждены теоретические расчеты длины кривой
        \end{itemize}
        
        \textbf{Практическое применение:}
        \begin{itemize}
            \item Образовательные цели в математике и информатике
            \item Компьютерная графика и моделирование природных объектов
            \item Исследование рекурсивных алгоритмов
            \item Основы фрактальной геометрии
        \end{itemize}
    }
    
    \block{Литература}{
        \begin{enumerate}
            \item Mandelbrot B. B. The Fractal Geometry of Nature. — 1982.
            \item Peitgen H.-O., Jürgens H., Saupe D. Chaos and Fractals. — 1992.
            \item Федер Е. Фракталы. — М.: Мир, 1991.
            \item Python Documentation. — 2024.
            \item Кроновер Р. М. Фракталы и хаос в динамических системах. — 2000.
        \end{enumerate}
    }
\end{columns}

\end{document}